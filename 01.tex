\documentclass[a4paper]{article}

\usepackage[T1]{fontenc}
% many useful symbols
\usepackage{textcomp}
\usepackage[english]{babel}
\usepackage{hyperref}
\usepackage{amsmath, amssymb, amsthm}
\usepackage{mathtools}
% for \lightning
\usepackage{stmaryrd}
\usepackage{geometry}
\usepackage{tikz-cd}

% Remove indentation globally
\setlength{\parindent}{0pt}
% Have blank lines between paragraphs
\usepackage[parfill]{parskip}

\hypersetup{
    colorlinks = true, % links instead of boxes
    urlcolor   = cyan, % external hyperlinks
    linkcolor  = blue, % internal links
    citecolor  = cyan   % citations
}

\newcommand{\R}{\mathbb{R}}
\newcommand{\C}{\mathbb{C}}
\newcommand{\Q}{\mathbb{Q}}
\newcommand{\N}{\mathbb{N}}
\newcommand{\A}{\mathbb{A}}
\newcommand{\Z}{\mathbb{Z}}
\renewcommand{\L}{\mathcal{L}}

\renewcommand{\div}{\operatorname{div}}
\renewcommand{\k}{\kappa}

% use bullets for items
\renewcommand{\labelitemii}{$\circ$}
\newcommand{\im}{\operatorname{im}}

\newcommand\numberthis{\addtocounter{equation}{1}\tag{\theequation}}

\newtheorem{theorem}{Theorem}[section]
\newtheorem{lemma}[theorem]{Lemma}

\theoremstyle{definition}
\newtheorem{definition}[theorem]{Definition}

\theoremstyle{definition}
\newtheorem{example}[theorem]{Example}

\theoremstyle{remark}
\newtheorem*{remark}{Remark}

\theoremstyle{definition}
\newtheorem{exercise}{Esercizio}[section]
\newtheorem*{exercise*}{Esercizio}

\title{Elliptic Curves - Homework 1}
\author{Francesco Minnocci}

\begin{document}
\maketitle

\textbf{1. }(in collaboration with Davide Pierrat)

Since $f\in \k(E)=\k(E_z)$ has no poles in the affine patch $E_z$, it lies in the intersection
\[
	\bigcap_{P \in E_z} \mathcal{O}_{E_z,P}= \mathcal{A}_{E_z}= \k[x,y]/(F_z),
\]
where $F_z=F(x,y,1)=y^2-x^3-Ax-B$. This implies that $f$ is of the form
\begin{equation}\label{eq:aff}
	g(x)+y\cdot h(x)
\end{equation} for some $g,h \in \k[x]$. Furthermore, through the identification seen in class between elements $\k(E_z)$ and quotients of homogeneous polynomials of the same degree, we can view $f$ as the polynomial
\[
	g\left(\frac{x}{z}\right)+y\cdot h\left(\frac{x}{z}\right).
\]

If moreover $f$ has no poles, then we look at the affine patch $E_y$, where the curve becomes
\[
	F_y=z-x^3-Axz^2-Bz^3.
\]
Then, we have seen that $x$ generates the local ring at $O=(0,1,0)$ (because $\partial_z{F_y}(O)\neq 0$), so we get $v_O{\left(\frac{x}{z}\right)}=-2$ as
\[
	z= \frac{x^3}{1-Axz-Bz^2},
\]
and thus $v_O{\left(\frac{1}{z}\right)}=-3$. Now, via the canonical isomorphism $\mathcal{O}_{E_z,O}\simeq \mathcal{O}_{E_y,O}$ the polynomial \eqref{eq:aff} corresponds to
\[
	g\left(\frac{x}{z}\right)+\frac{1}{z}\cdot h\left(\frac{x}{z}\right),
\]
and comparing the parity of the valuation at $O$ of the two summands we deduce that $f$ must be constant.


\textbf{2.} Throughout this exercise we will tacitly use the fact that, for a smooth projective plane curve $X$,
\[\deg(\div(f))=0\] holds for any $f\in \k(X)^\times$.

Suppose first $D=\div(g)$ is the divisor of a function. Then,
\[
	\L(D)=\{f\mid\div(f)+\div(g)\geq 0\},
\]
so we are looking for those functions $f$ such that $v_P(g)+v_P(f)\geq 0$ for all $P\in X$, and since
\[
	\sum_P{v_P(g)}=\sum_P{v_P(f)}=0,
\]
this means that $v_P(g)=-v_P(f)$ for all $P\in X$. In other words,
\[
	v_P\left(\frac{g}{f}\right)=0 ~~ \forall P \in X \implies \frac{g}{f}\in\k^\times,
\]
where the last implication follows from the fact that
\[
	\bigcap_P{\mathcal{O}_{X,P}}=\k
\]
for any smooth projective plane curve $X$. This implies that $\dim(\L(D))=1$, as we have shown
\[
	\L(D)=\{\lambda \cdot \frac{1}{f}\mid\lambda\in\k\}
\]

Let us now proceed by contrapositive: assume that the dimension of $\dim{\L(D)}$ is strictly positive, where $D=\sum_P{m_P P}$. Then, there must be a function $f\in\k(X)^\times$ such that
\[
	v_P(f)\geq -m_P
\]
for all $P\in X$, and by hypothesis we have
\[
	\sum_P{m_P}=\sum_P{v_P(f)}=0.
\]
Therefore, as in the above argument we deduce $v_P(f)=-m_P$ for all $p\in X$, which shows that $D$ is the divisor of the function $\frac{1}{f}\in \k(X)^\times$.

\textbf{3.} We begin by noting that the conic
\[
	C: x^2+y^2=z^2
\]
defined over the finite field $\mathbb{F}_q$ is isomorphic to the projective line $\mathbb{P}^1=\mathbb{P}^1_{\mathbb{F}_q}$, via the morphism
\begin{align*}
	\mathbb{P}^1 & \longrightarrow C                              \\
	[x_0:x_1]    & \longmapsto [x_0^2-x_1^2:2x_0x_1:x_0^2+x_1^2],
\end{align*}
whose inverse is given by the morphism
\begin{align*}
	C       & \longrightarrow \mathbb{P}^1                \\
	[x:y:z] & \longmapsto \begin{cases}
		                      [y:x+z]\text{ if } y(z+x)\neq 0 \\
		                      [z-x:y]\text{ if } y(z-x)\neq 0 \\
	                      \end{cases}
\end{align*}
Therefore, if $q=p^m$ the number of $\mathbb{F}_q$-points on $C$ is equal to
\[
	N_m=\left|\mathbb{P}^1\right|=1+q^m
	,\]
and we can compute its zeta function as
\begin{align*}
	Z_C(t) & =\exp{\left(\sum_{m\geq 1}\frac{t^m}{m}(1+q^m)\right)}                                           \\
	       & =\exp{\left(\sum_{m\geq 1}\frac{t^m}{m}\right)}\exp{\left(\sum_{m\geq 1}\frac{(qt)^m}{m}\right)} \\
	       & = \frac{1}{(1-t)(1-qt)},
\end{align*}
where the last equality follows from the well-known logarithmic expansion
\[
	\log\left(\frac{1}{1-x}\right)=\sum_{m\geq 1}\frac{x^m}{m}
\]
Finally, we prove the functional equation for $C$:
\begin{align*}
	Z_C(t)=\frac{1}{1-t}\cdot \frac{1}{1-qt} & =\frac{1}{qt^2}\left(\frac{t}{1-t}\right)\left(\frac{qt}{1-qt}\right)                  \\
	                                         & =q^{-1}t^{-2}\left(\frac{1}{1-\frac{1}{t}}\right)\left(\frac{1}{1-\frac{1}{qt}}\right) \\
	                                         & = q^{g-1}t^{g-2}Z_C\left(\frac{1}{qt}\right),
\end{align*}
which concludes since $C$ has genus
\[
	g= \frac{(2-1)(2-2)}{2}=0.
\]

\end{document}
