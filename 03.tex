\documentclass[a4paper]{article}

\usepackage[T1]{fontenc}
% many useful symbols
\usepackage{textcomp}
\usepackage[english]{babel}
\usepackage{hyperref}
\usepackage{amsmath, amssymb, amsthm}
\usepackage{mathtools}
% for \lightning
\usepackage{stmaryrd}
\usepackage{geometry}
\usepackage{tikz-cd}
\usepackage{cancel}

\hypersetup{
    colorlinks = true, % links instead of boxes
    urlcolor   = cyan, % external hyperlinks
    linkcolor  = blue, % internal links
    citecolor  = cyan   % citations
}

\newcommand{\R}{\mathbb{R}}
\newcommand{\C}{\mathbb{C}}
\newcommand{\Q}{\mathbb{Q}}
\newcommand{\N}{\mathbb{N}}
\newcommand{\A}{\mathbb{A}}
\newcommand{\Z}{\mathbb{Z}}
\renewcommand{\L}{\mathcal{L}}
\newcommand{\Qp}{\Q_p}
\newcommand{\Fp}{\mathbb{F}_p}
\newcommand{\Fq}{\mathbb{F}_q}
\newcommand{\Fpp}{\overline{\mathbb{F}}_p}
\newcommand{\Fqq}{\overline{\mathbb{F}}_q}

% Sha for the Tate-Shafarevich group
\DeclareFontFamily{U}{wncy}{}
\DeclareFontShape{U}{wncy}{m}{n}{<->wncyr10}{}
\DeclareSymbolFont{mcy}{U}{wncy}{m}{n}
\DeclareMathSymbol{\Sh}{\mathord}{mcy}{"58}

\newcommand{\E}{\overline{E}}

\renewcommand{\div}{\operatorname{div}}
\renewcommand{\k}{\kappa}

% Display math
\newcommand{\ssfrac}[2]{
        \raisebox{+0.3ex}{$#1$}
        /
        \raisebox{-0.3ex}{$#2$}
}
% Inline math
\newcommand{\sfrac}[2]{
        \raisebox{+0.3ex}{\scalebox{0.9}{$#1$}}
        /
        \raisebox{-0.3ex}{\scalebox{0.9}{$#2$}}
}

% use bullets for items
\renewcommand{\labelitemii}{$\circ$}
\newcommand{\im}{\operatorname{im}}

\newcommand\numberthis{\addtocounter{equation}{1}\tag{\theequation}}

\newtheorem{theorem}{Theorem}[section]
\newtheorem{lemma}[theorem]{Lemma}

\theoremstyle{definition}
\newtheorem{definition}[theorem]{Definition}

\theoremstyle{definition}
\newtheorem{example}[theorem]{Example}

\theoremstyle{remark}
\newtheorem*{remark}{Remark}

\theoremstyle{definition}
\newtheorem{exercise}{Esercizio}[section]
\newtheorem*{exercise*}{Esercizio}

\title{Elliptic Curves - Homework 3}
\author{Francesco Minnocci}

\begin{document}
\maketitle

\textbf{8.}
To find an isomorphism between
$C: x^2+y^2+z^2=0$ and $\mathbb{P}^1_{\overline{\mathbb{Q}}}$, we first observe that $C$ is isomorphic to $C^\prime: x^2++y^2=z^2$ via
\begin{align*}
        C^\prime & \to C             \\
        [x:y:z]  & \mapsto [x:y:iz],
\end{align*}
with inverse $[x:y:z] \mapsto [x:y:-iz]$. Furthermore, $C^\prime$ is isomorphic to $\mathbb{P}^1_{\overline{\mathbb{Q}}}$ via the map constructed in the first homework,and we can compose the two isomorphisms to get
\begin{align*}
        \varphi:\mathbb{P}^1_{\overline{\mathbb{Q}}} & \to C                             \\
        [x:y]                                        & \mapsto [y^2-x^2:2xy:i(x^2+y^2)],
\end{align*}
whose inverse is given by the morphism
\begin{align*}
        \varphi^{-1}:C & \longrightarrow \mathbb{P}^1_{\overline{\mathbb{Q}}} \\
        [x:y:z]        & \longmapsto \begin{cases}
                                             [y:x-iz]\text{ if } y(x-iz)\neq 0   \\
                                             [-x-iz:y]\text{ if } y(-x-iz)\neq 0 \\
                                     \end{cases}
\end{align*}
We have thus found an isomorphism between $C$ and $\mathbb{P}^1_{\overline{\mathbb{Q}}}$ over $\overline{\mathbb{Q}}$, which shows that $C$ is a twisted form of $\mathbb{P}^1_\mathbb{Q}$. To find a corresponding cocycle, notice that $\varphi$ is already defined over $\mathbb{Q}(i)$. This means that the action of $\operatorname{Gal}(\overline{\mathbb{Q}}/\mathbb{Q})$ on the group $\operatorname{Aut}(\mathbb{P}^1_{\overline{\mathbb{Q}}})$ factors through $\operatorname{Gal}(\mathbb{Q}(i)/\mathbb{Q})$, and so
for $\sigma \in \operatorname{Gal}(\overline{\mathbb{Q}}/\mathbb{Q})$ we have that
\[
        \varphi^{-1}\circ \sigma(\varphi)=\varphi^{-1}\circ(\sigma\circ\varphi\circ\sigma^{-1}): \mathbb{P}^1_{\overline{\mathbb{Q}}}  \longrightarrow \mathbb{P}^1_{\overline{\mathbb{Q}}}
\]
sends
\begin{align*}
        [x:y] & \longmapsto \begin{cases}
                                    [x:y]  & \text{ if }   \sigma(i)=i  \\
                                    [-y:x] & \text{ if }  \sigma(i)=-i,
                            \end{cases}
\end{align*}
which can be checked directly by the explicit formulas for $\varphi$ and $\varphi^{-1}$.

\textbf{9.} In the following, we denote by $\E$ the $\Fqq$-points of the elliptic curve $E$.

\noindent Taking the exact sequence
\begin{equation*}
        0 \to \E[\ell] \to \E \to \E \to 0
\end{equation*}
yields a following exact sequence in cohomology:
\[
        \begin{tikzcd}
                \cdots \ar[r] & E(\Fq) \ar[r, "\ell"] &E(\Fq) \ar[r] &H^1(\Fq, \E[\ell]) \ar[r] &\cancel{H^1(\Fq, \E)} \ar[r] &\cdots
        \end{tikzcd}
\]
where the last term is zero by Lemma 12.12 in the notes. In others words, the group $H^1(\Fq, \E[\ell])$ is a quotient of the finite abelian group $E(\Fq)$, hence finite itself.

Finally, for all but finitely many $\ell$ (all except those dividing the order of $E(\Fq)$), multiplication by $\ell$ is an automorphism of the group $E(\Fq)$, which in particular has trivial cokernel $H^1(\Fq, \E[\ell])$. This shows that the group $H^1(\Fq, \E[\ell])$ is trivial for all but finitely many $\ell$.

\textbf{10.} (in collaboration with Marco Sanna)
\noindent Let $\overline{K}$ be a fixed algebraic closure of $K$. By the N\'eron-Ogg-Shafarevich criterion, the curve $E$ has good reduction if and only the whole Tate module is
fixed by the inertia subgroup $I$.

In the case of bad reduction, we will show that $E$ cannot have additive reduction over $K^{\text{nr}}$; this implies that it has either good reduction over $K^{\text{nr}}$ (and we have seen that in
this case $E$ has good reduction over $K$) or multiplicative reduction over $K^{\text{nr}}$. In this last case, $E$ must also have multiplicative reduction over $K$ (which is our goal); indeed, if the $E$ has equation $y^2 = x^3 + Ax + B$ over $K$ with discriminant $\Delta$, assume by contradiction that $E$ has additive reduction over $K$ (that is, up to translation $\overline{A} = \overline{B} = 0$). Then, since $E$ has multiplicative reduction over $K^{\text{nr}}$, the valuation $v_{K^{\text{nr}}}(\Delta)$ cannot be minimal, and so there is some coordinate change
\[
        x = u^2 x', \quad y = u^3 y'
\]
sends $\Delta$ to $u^{-12}\Delta$, and
\[
        v_{K^{\text{nr}}}(u^{-12}\Delta) < v_{K^{\text{nr}}}(\Delta)=v_K(\Delta).
\]
Now, after fixing a uniformizer $\pi$ of
$\mathcal{O}_K$ (which is also a uniformizer of $\mathcal{O}_{K^{\text{nr}}}$), we can write $u$ as $w\cdot \pi^r$ for some $w \in \mathcal{O}_{K^{\text{nr}}}$ and $r \in \Z$. Then, we have
we can write $u$ as $w\cdot \pi^r$, and the coordinate change
\[
        x \to \pi^{2r}x, \quad y \to \pi^{3r}y
\]
sends $\Delta$ to $\pi^{-12r}\Delta$, which contradicts the minimality of $v_K(\Delta)$:
\[
        v_K(\pi^{-12r}\Delta) = v_{K^{\text{nr}}}(u^{-12}\Delta) < v_K(\Delta).
\]
Suppose now that $E$ has additive reduction over
$K^{\text{nr}}$, and let $\kappa$ be the residue field of $K^{\text{nr}}$. As in the proof of Theorem 13.4, we first choose $m$ large enough such that
$$l^m> |\sfrac{E(K^{\text{nr}})}{E(K^{\text{nr}})^{(0)}}|.$$
By assumption, there is some non-trivial element $(P_i)$ of $T_\ell(E)$ which is fixed by $I$. Setting
\[
        \overline{n}\coloneqq \min\{n:\,\pi_n((P_i))\neq O\},
\]
by definition of $T_\ell(E)$ this means that
\[Q:=\pi_{\overline{n}+m-1}((P_i))\]
has order $\ell^m$, and since $Q$ is fixed by $I$, it is defined over $K^{\text{nr}}$. By our choice of $m$, the order $\ell$ subgroup generated by $(l^{m-1}Q)$ must be contained in $E(K^{\text{nr}})^{(0)}[l]$, but since $E(K^{\text{nr}})^{(1)}$ has no $\ell$-torsion, the reduction map $r$ is an injection on $E(K^{\text{nr}})^{(0)}[l]$, which should send $(l^{m-1}Q)$ to a point of order $\ell$ in $\overline{E}(\kappa)_\text{smooth}\simeq \kappa^{+}\simeq \Fpp^{+}$; this yields a contradiction, since $\Fpp^{+}$ has no points of order $\ell$ for $\ell\neq p$.

\end{document}
